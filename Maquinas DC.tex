%%%%%%%%%%%%%%%%%%%%%%%%%%%%%%%%%%%%%%%%%%%%%%%%%%%%%%%%%%%%%%
%%%%		Informe de Máquinas DC
%%%%	Fecha	: Septiembre 2020
%%%%	Autor	: Luis Millan
%%%%	Correo	: lmillan131@gmail.com
%%%%
%%%%%%%%%%%%%%%%%%%%%%%%%%%%%%%%%%%%%%%%%%%%%%%%%%%%%%%%%%%%%%

\documentclass[11pt,letterpaper]{article}
\usepackage[activeacute,spanish]{babel}
\usepackage[utf8]{inputenc}
\usepackage[letterpaper,includeheadfoot, top=0.5cm, bottom=3.0cm, right=2.0cm, left=2.0cm]{geometry}
\renewcommand{\familydefault}{\sfdefault}
\usepackage{graphicx}
\usepackage{color}
\usepackage{hyperref}
\usepackage{amssymb}
\usepackage{url}
\usepackage{fancyhdr}
\usepackage{hyperref}
\usepackage{subfig}
\usepackage{listings}
\usepackage{mathrsfs,amsmath}
\lstset{language=C, tabsize=4,framexleftmargin=5mm,breaklines=true}

% =============== Inicio de Documento =============== 
\begin{document}
% =============== Portada =============== 
\newpage
\pagestyle{fancy}
\fancyhf{}
%\fancyhead[L]{ \includegraphics[scale=0.9]{logodgf.jpeg} }
\vspace*{6cm}
\begin{center}
\Huge  {Trabajo de Investigación}\\
\Huge  {Campo Magnetico Rotatorio}\\
\Huge  {Maquinas Asincronicas}\\
\vspace{1cm}
\end{center}
\vfill
\begin{flushright}
\begin{tabular}{ll}
Autor: & Luis E. Millán U.\\
Profesor: & Ing. Hector Delgado\\
& \today\\
& Caracas, Venezuela.
\end{tabular}
\end{flushright}

% =============== Encabezado y pie de Pagina ===============
\newpage
\pagestyle{fancy}
\fancyhf{}
%Encabezado
\fancyhead[L]{\rightmark}
\fancyhead[L]{\small \rm \textit{Sección \rightmark}}
\fancyhead[R]{\small \rm \textbf{\thepage}}
%Pie 
\fancyfoot[L]{\small \rm \textit{Br. L. Millán}}
\fancyfoot[R]{\small \rm \textit{Maquinas Asincronicas}}
\renewcommand{\sectionmark}[1]{\markright{\thesection.\ #1}}
\renewcommand{\headrulewidth}{0.5pt}
\renewcommand{\footrulewidth}{0.5pt}
\tableofcontents
%\listoffigures
%==========================================%
%  MOTORES
%==========================================%
\newpage
\section{Objetivos}
\section{Instrumentos y Equipos}
\section{Condiciones de Ensayo}
\section{Procedimiento}
\section{Disgramas}
\section{Desempeño}
\section{Resultados}
\subsection{Caídas Internas de Tensión}
	Debido a que las resistnecias son medidas en temperatura ambiente (25$^{\circ}C$) dichas mediciones deben ser convertidas a una temperatura de referencia (75$^{\circ}C$) mediante la ecuación \ref{ec:refResistenciasTemp}.
	Para el cálculo de la incertidumbre se utiliza la ecuación \ref{ec: incerResistCampo}.\\
	
	Los resultados luego de referenciar las diferentes resistencias a la temperatura correcta, incluir la incertidumbre y obtener un promedio son ilustrados en las tablas \ref{tab:resistencia de campo}, \ref{tab:resistencia serie} y \ref{tab:resistencia de Armadura}, para las resistencias de campo, serie y armadura.\\
\begin{equation}\label{ec:refResistenciasTemp}
	R_{r} = R_{medida} \cdot \frac{T_{r}+T_{k}}{T_{m}+T_{k}}
\end{equation}
Donde:
\begin{itemize}
	\item $R_{r}$ = Resistencia de la temperatura deseada
	\item $R_{m}$ = Resistencia medida a la temperatura $T_{m}$
	\item $T_{m}$ = Temperatura del Laboratorio
	\item $T_{r}$ = Temperatura de Referencia
	\item $T_{k}$ = $234.5 ^{\circ} C$ 
\end{itemize}

\begin{equation}\label{ec:ohm}
	R = \frac{V}{I}
\end{equation}

\begin{equation}\label{ec: incerResistCampo}
	\bigtriangleup R = \frac{\delta V}{\delta R} \cdot \bigtriangleup V + \frac{\delta I}{\delta R} \cdot \bigtriangleup I
\end{equation}

\begin{table}[t]
	\begin{center}
		\begin{tabular}{| r | l | c |}
			\hline
			$V_{DC}$ $[V]$ & $I_{DC}$ $[A]$ & $R_{F}$ $[\Omega]$ \\ \hline
			$63 \pm 1$ & $1.00 \pm 0.02$ & $75.14 \pm 2.26 $ \\
			$58 \pm 1$ & $0.92 \pm 0.02$ & $75.19 \pm 2.46$ \\
			$48 \pm 1$ & $0.76 \pm 0.02$ & $75.32 \pm 2.97$ \\ \hline
			- & Promedio & $75.21 \pm 2.56$ \\ \hline
		\end{tabular}
		\caption{Resistencia de Campo}
		\label{tab:resistencia de campo}
	\end{center}
\end{table}

\begin{table}[t]
	\begin{center}
		\begin{tabular}{| r | l | c |}
			\hline
			$V_{DC}$ $[V]$ & $I_{DC}$ $[A]$ & $R_{S}$ $[\Omega]$ \\ \hline
			$60 \pm 1$ & $3.3 \pm 0.1$ & $21.69 \pm 0.85$ \\
			$72 \pm 1$ & $4.0 \pm 0.1$ & $21.47 \pm 0.70$ \\
			$78 \pm 1$ & $4.6 \pm 0.1$ & $20.22 \pm 0.59$ \\  \hline
			- & Promedio & $21.13 \pm 0.71$ \\ \hline
		\end{tabular}
		\caption{Resistencia Serie}
		\label{tab:resistencia serie}
	\end{center}
\end{table}

\begin{table}[t]
	\begin{center}
		\begin{tabular}{| r | l | c | c |}
			\hline
			$V_{DC}$ $[V]$ & $V_{R_{shunt}}$ $[mV]$ & $I_{DC}$ $[A]$ & $R_{A}$ $[\Omega]$ \\ \hline
			$1.65 \pm 0.05$ & $18.0 \pm 0.1$ & $3.60 \pm 0.02$ & $0.46 \pm 0.02$ \\
			$1.50 \pm 0.05$ & $16.0 \pm 0.1$ & $3.20 \pm 0.02$ & $0.47 \pm 0.02$ \\
			$1.15 \pm 0.05$ & $12.5 \pm 0.1$ & $2.50 \pm 0.02$ & $0.46 \pm 0.02$ \\  
			$0.75 \pm 0.05$ & $8.0 \pm 0.2$  & $1.60 \pm 0.04$ & $0.47 \pm 0.04$ \\  \hline
			- & - & Promedio & $0.47 \pm 0.03$ \\ \hline
		\end{tabular}
		\caption{Resistencia de Armadura}
		\label{tab:resistencia de Armadura}
	\end{center}
\end{table}

\subsection{Curva de Vacío}
\begin{table}[t]
	\begin{center}
		\begin{tabular}{| r | l | c | c |}
			\hline
			$\%$Tensión  & Corriente de Campo [$I_{F}$]  & $E_{A}$ Subida [V] & $E_{A}$ Bajada [V] \\ \hline
			$0.00 U_{cc}$ & $0.00$          & $8   \pm 1$ & $12  \pm 1$ \\
			$0.25 U_{cc}$ & $0.16 \pm 0.16$ & $42  \pm 1$ & $45  \pm 1$ \\
			$0.50 U_{cc}$ & $0.50 \pm 0.1$  & $58  \pm 1$ & $58  \pm 1$ \\
			$0.75 U_{cc}$ & $0.70 \pm 0.1$  & $74  \pm 1$ & $74  \pm 1$ \\
			$1.00 U_{cc}$ & $0.90 \pm 0.1$  & $100 \pm 1$ & $100 \pm 1$ \\  
			$1.12 U_{cc}$ & $1.10 \pm 0.1$  & $110 \pm 1$ & $110 \pm 1$ \\
			$1.25 U_{cc}$ & $1.30 \pm 0.1$  & $12  \pm 1$ & $120 \pm 1$ \\  \hline
		\end{tabular}
		\caption{Curva de Vacio}
		\label{tab:Curva de Vacio}
	\end{center}
\end{table}
\section{Análisis de los Resultados}
\section{Conclusiones}
\section{Hoja de Datos}
\end{document}